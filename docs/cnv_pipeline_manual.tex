\documentclass[11pt]{article}

\usepackage{fullpage,times}
\usepackage{graphicx}
\usepackage{amsmath}
\usepackage{siunitx}
\usepackage{float}

\usepackage[sort]{natbib}

\usepackage{hyperref}
\hypersetup{
  colorlinks=true,
  allcolors=blue
}

\newcommand{\prog}[1]{\texttt{#1}}

\title{A Copy Number Analysis Pipeline}
\author{Rish Prabakar}

\begin{document}
\maketitle

%%%%%%%%%%%%%%%%%%%%%%%%%%%%%%%%%%%%%%%%%%%%%%%%%%%%%%%%%%%%%%%%%%%%%%%%%%%%%%%%
%%%%%%%%%%%%%%%%%%%%%%%%%%%%%%%%%%%%%%%%%%%%%%%%%%%%%%%%%%%%%%%%%%%%%%%%%%%%%%%%
%%%%%%%%%%%%%%%%%%%%%%%%%%%%%%%%%%%%%%%%%%%%%%%%%%%%%%%%%%%%%%%%%%%%%%%%%%%%%%%%
\section{Introduction}
This document is intended to serve as a walk through of the copy number
analysis pipeline.

\paragraph{Assumptions:}
\begin{enumerate}
  \item A familiarity with the UNIX computing environment.
  \item The ability to compile and install tools and packages.
  \item A familiarity with the commonly used bioinformatic file types such
    as fasta, fastq, and sam/bam.
\end{enumerate}

A personal workstation with at least 16GB of memory and 5GB of free disk
space can be used for the analysis of one or at most a few samples.
%
For processing a large number of samples, it is highly recommended that
the analysis is done on a compute cluster. A compute cluster provides
hundreds of nodes that can process files in parallel, and thus
significantly speeding up the analysis.


%%%%%%%%%%%%%%%%%%%%%%%%%%%%%%%%%%%%%%%%%%%%%%%%%%%%%%%%%%%%%%%%%%%%%%%%%%%%%%%%
%%%%%%%%%%%%%%%%%%%%%%%%%%%%%%%%%%%%%%%%%%%%%%%%%%%%%%%%%%%%%%%%%%%%%%%%%%%%%%%%
%%%%%%%%%%%%%%%%%%%%%%%%%%%%%%%%%%%%%%%%%%%%%%%%%%%%%%%%%%%%%%%%%%%%%%%%%%%%%%%%
\section{Copy number analysis}
% CNV pipeline background
This copy number analysis pipeline is based on the procedure described in
\citep{baslan2012genome,kendall2014computational}, with modifications to
bin boundaries that are designed for 150bp paired-end reads, and with
an additional step of explicitly filtering reads that align to ambiguous
or problematic regions of the reference genome. Briefly, the human
reference genome (hg19 or hg38) is split into bins (typically 5000, but
can higher or lower depending on the resolution needed and number of
reads sequenced) containing an equal number of uniquely mappable
locations, and the bin counts are determined using uniquely mapped reads
that do not align to the ambiguous or problematic regions. Bins with
spuriously high counts (``bad bins,'' typically around centromeric and
telomeric regions) are masked for downstream analysis. This procedure
normalizes bin counts for biases correlated with GC content by fitting a
LOWESS curve to the GC content by bin count, and subtracting the LOWESS
estimate from each bin.  Circular binary segmentation (CBS)
\citep{olshen2004circular}, then identifies breakpoints in the
normalized bin counts.


%%%%%%%%%%%%%%%%%%%%%%%%%%%%%%%%%%%%%%%%%%%%%%%%%%%%%%%%%%%%%%%%%%%%%%%%%%%%%%%%
\subsection{Tools and packages required}
The following tools and packages are required for copy number analysis:
\begin{enumerate}
\setlength{\itemsep}{0pt}
  \item \href{https://www.bioinformatics.babraham.ac.uk/projects/fastqc/}
    {FastQC}
  \item \href{https://github.com/lh3/bwa}{bwa}
  \item \href{http://www.htslib.org/}{samtools}
  \item \href{https://broadinstitute.github.io/picard/}{picard tools}
    (requires Java, unfortunately)
  \item Python ($\geq$ 3.0)
  \item Python packages:
    \href{https://pysam.readthedocs.io/en/latest/}{pysam}
  \item R ($\geq 4.0$)
  \item R packages:
    \href{https://cran.r-project.org/web/packages/optparse/index.html}{optparse}
    and \href{https://bioconductor.org/packages/release/bioc/html/DNAcopy.html}
    {DNAcopy}
  \item Optional: \href{https://multiqc.info/}{MultiQC}
\end{enumerate}

All tools are assumed to the located in a directory specified in
\texttt{\$PATH} and all files are assumed to be in the current working
directory. If they are not, the commands below can be modified to
include the absolute or relative paths to the tool and required files.

%%%%%%%%%%%%%%%%%%%%%%%%%%%%%%%%%%%%%%%%%%%%%%%%%%%%%%%%%%%%%%%%%%%%%%%%%%%%%%%%
\subsection{Preparing the reference genome}
\label{index}
\paragraph{hg19 download and pre-processing:}
The hg19 reference genome can be downloaded from the
\href{http://hgdownload.cse.ucsc.edu/goldenpath/hg19/bigZips/chromFa.tar.gz}
{UCSC Genome Browser}. Alternate haplotypes are not used for analysis.
Additionally, the
\href{https://en.wikipedia.org/wiki/Pseudoautosomal\_region}{pseudoautosomal
regions} on chrY are masked by replacing the bases with Ns. All the
chromosome files can then the merged into one fasta file.

\paragraph{hg38 download:}
The hg38 reference genome can be downloaded from the 
\href{https://hgdownload.soe.ucsc.edu/goldenPath/hg38/bigZips/analysisSet/hg38.analysisSet.fa.gz}
{UCSC Genome Browser}. 
The hg38 ``Analysis set'' has the PAR regions masked and does not
contain alternate haplotypes. No pre-processing is required and the genome
can be used directly. 

\paragraph{Generating BWA index:}
% build the index
\prog{bwa}, the program used to align reads to the reference genome,
requires the reference genome to be prepossessed once before the first
use.
\begin{verbatim}
$ bwa index {reference.fa}
\end{verbatim}

% prebuilt index
% The reference genome and the prebuilt index can be downloaded from
% \href{}{here}. It is highly recommended to use this pre-built index % to
% maintain consistency.

%%%%%%%%%%%%%%%%%%%%%%%%%%%%%%%%%%%%%%%%%%%%%%%%%%%%%%%%%%%%%%%%%%%%%%%%%%%%%%%%
\subsection{Pre-alignment QC}
Prior to any analysis it is crucial to QC the fastq files to detect any
errors in the library preparation or the sequencing process.
\prog{FastQC} provides a set of diagnostic plots on fastq files.
% cmd
\begin{verbatim}
$ fastqc read_1.fastq.gz read_2.fastq.gz -o fastqc_outdir
\end{verbatim}

% output
The output of \prog{FastQC} is one HTML file per fastq file containing
some basic statistics and a set of diagnostic plots.
These HTML files can be view on any browser. An explanation of these
plots are provided
\href{https://www.bioinformatics.babraham.ac.uk/projects/fastqc/}{here}.
Some important metrics to look for are the read length distribution,
sequence quality, over-represented sequences such as sequencing
adapters, and GC content.

% muliqc
It is generally inconvenient to view one HTML file for every fastq
file. The tool \prog{MultiQC} can be used to aggregate all the
information for all the samples into one HTML file.

%%%%%%%%%%%%%%%%%%%%%%%%%%%%%%%%%%%%%%%%%%%%%%%%%%%%%%%%%%%%%%%%%%%%%%%%%%%%%%%%
\subsection{Align reads to the reference genome}
The first step in processing the reads is to align them to the reference
genome to determine their location on the reference genome.
The reads are aligned to the reference genome using \prog{bwa}
\citep{li2013aligning}.
The input to \prog{bwa} are a pair of fastq files and the pre-built
reference index (as described in section \ref{index}).

The input reads usually contain primer sequences from whole genome
amplification or adapter sequences from the Illumina library preparation
process at the ends. \prog{bwa} performs a local alignment on the reads
and soft clips any of these unwanted technical bases that would not
align to the reference genome. Therefore, is not required to explicitly
trim the adapter sequences prior to processing with \prog{bwa}.

% cmd
\begin{verbatim}
$ bwa mem -t {threads} {path_to_bwa_index} \
    read_1.fastq.gz read_2.fastq.gz \
    1> sample.sam 2> sample_bwa_log.txt
\end{verbatim}

% output
The output of \prog{bwa} is a sam file. The first few lines of a sam
file contain a header that starts with \texttt{@}. The subsequent
lines contain one entry for each read, and thus two entries for a read
pair (but could contain more than two entries when a read is split and
aligned to more than one location on the reference).
%
Each entry is tab delimited, and the first five columns provide the
most useful information in the context of copy number profiling: (1) The
read name. (2) A decimal representation of a bitwise combination of 12
bits, where each bit represents a property of the alignment. (3) The
chromosome to which the read aligns to. (4) The position on the
chromosome to which the read aligns to. (5) Mapping quality of the read
which is calculated as $-10 \log_{10} \mathrm{Pr(mapping\ position\ is
\ wrong)}$.
%
A detailed specification of a sam file format can be found
\href{https://samtools.github.io/hts-specs/SAMv1.pdf}{here}.

% convert to bam
Sam files are convenient to view on a text editor. However, these files
consume a lot of disk space since they are not compressed. Bam files are
the compressed version of sam files, and the formats can be converted
with \prog{samtools} (the sam file can then be deleted).
\begin{verbatim}
$ samtools view -@ {threads} -b -o sample.bam sample.sam
\end{verbatim}

%%%%%%%%%%%%%%%%%%%%%%%%%%%%%%%%%%%%%%%%%%%%%%%%%%%%%%%%%%%%%%%%%%%%%%%%%%%%%%%%
\subsection{Remove PCR duplicates}
There are several steps in the process of going from
$\sim$6\si{\pico\gram} of DNA in a cell to the $\sim$100\si{\nano\gram}
required for Illumina library preparation involves PCR amplification,
which could lead to capturing two identical DNA fragments in the
sequencing process.  Since these reads are identical, the presence of
more than one read pair for identical fragments does not provide any
additional information, but could lead to a bias in the downstream
analysis.

% How PCR duplicates are removed
PCR duplicates are removed based on the assumption that two read pairs
that align to exactly the same location on the reference genome are more
likely due to a PCR duplicate rather than two different fragments. This
is a reasonable assumption for whole genome sequencing at a low coverage
especially for a single cell. (However, this method of removing PCR
duplicates cannot be used for high coverage or targeted sequencing.)
Multiple read pairs that align to the reference at the same 5' location
for both the forward and the reverse reads are considered as PCR
duplicates. A graphical representation of a PCR duplicate can be found
\href{https://www.htslib.org/algorithms/duplicate.html}{here}.

% A good library
A good library that has a high complexity contains very few duplicated
reads. A library that contains a high fraction of PCR duplicates ($>30-40\%$)
could be indicative of low DNA content in the cell or high genome drop out,
and should be treated with caution.

% cmd
PCR duplicates are removed with \prog{samtools} with a set of four
commands. The first command sorts the alignments in a bam file based on
the read names so that all the alignments from the same read pair are
adjacent in the bam file.
\begin{verbatim}
$ samtools collate -@ {threads} -o sample_collate.bam sample.bam
\end{verbatim}
%
\prog{bwa} and other mapping tools typically do not provide the correct
information about the insert size for the read pairs. \prog{samtools
fixmate} fixes these.
\begin{verbatim}
$ samtools fixmate -@ {threads} -m sample_collate.bam \
    sample_fixmate.bam
\end{verbatim}
%
The bam file is then sorted based on the genome coordinates to
facilitate removing PCR duplicates.
\begin{verbatim}
$ samtools sort -@ {threads} -o sample_sorted.bam sample_fixmate.bam
\end{verbatim}
%
Finally, the PCR duplicates are removed and only the best alignment for
each set of duplicates is retained.
\begin{verbatim}
$ samtools markdup -@ {threads} -r sample_sorted.bam sample_rmdup.bam
\end{verbatim}

% output
The final output of these steps is a bam file in which all the duplicate
reads are discarded, and thus contains fewer alignments than the input bam
file.


%%%%%%%%%%%%%%%%%%%%%%%%%%%%%%%%%%%%%%%%%%%%%%%%%%%%%%%%%%%%%%%%%%%%%%%%%%%%%%%%
\subsection{Remove low quality and ambiguously mapped reads}
Reads that align to several different locations on the reference
genome receive a low mapping quality score (in the fifth column of a
sam entry). Since a unique location for these alignments cannot be
determined, they do not provide any useful information for copy number
profiling and they can be discarded. Further, for reads that get split
and align to multiple locations, only the best alignment is retained. The
alignments are filtered with \prog{samtools}:
% cmd
\begin{verbatim}
$ samtools view -@ {threads} -q 30 -F 0x800 -o sample_unique.bam \
    sample_rmdup.bam
\end{verbatim}
% output
The output is a bam file that retains only the unique alignments.

%%%%%%%%%%%%%%%%%%%%%%%%%%%%%%%%%%%%%%%%%%%%%%%%%%%%%%%%%%%%%%%%%%%%%%%%%%%%%%%%
\subsection{Remove mates}
Since a sam file contains two entries for each read pair, one of these
needs to be removed prior to copy number profiling. Not doing so would
lead to double counting each alignment. \prog{samtools} is used to
remove the alignments corresponding to the second read of the read pair.
% cmd
\begin{verbatim}
$ samtools view -@ {threads} -f 0x40 -h -o sample_fwd.bam \
    sample_unique.bam
\end{verbatim}
% output
The output is a bam file that retains only the alignments that belong to
the first read of the read pair.

%%%%%%%%%%%%%%%%%%%%%%%%%%%%%%%%%%%%%%%%%%%%%%%%%%%%%%%%%%%%%%%%%%%%%%%%%%%%%%%%
\subsection{Post alignment QC}
It is highly recommended to check the number of alignments that are
discarded after each step of the above filtering process. The number of
alignments, the number of read pairs, etc in a sam file can be determined
using the command:
% cmd
\begin{verbatim}
$ samtools flagstat {bam_file.bam}
\end{verbatim}
Of note, in the output of \prog{samtools flagstat}, the fraction of
aligned reads are calculated based on the number of alignments in the sam
file (and not based on the number of reads in the fastq files for the
sample).


%%%%%%%%%%%%%%%%%%%%%%%%%%%%%%%%%%%%%%%%%%%%%%%%%%%%%%%%%%%%%%%%%%%%%%%%%%%%%%%%
\subsection{Insert size distribution}
As an additional QC, it is important to check the insert size
distribution of the aligned reads. A good sequencing library should have
a majority of reads with an insert size greater than 100bp. Samples
with a large fraction of inserts less than 50bp should be treated with
caution.
% cmd
\begin{verbatim}
$ java -jar picard CollectInsertSizeMetrics I=sample_unique.bam \
  O=sample_insert_sz.txt H=sample_insert_sz.pdf
\end{verbatim}

%%%%%%%%%%%%%%%%%%%%%%%%%%%%%%%%%%%%%%%%%%%%%%%%%%%%%%%%%%%%%%%%%%%%%%%%%%%%%%%%
\subsection{Generate bin counts}
At this stage, the sam file is ready to be used for copy number
profiling.  The next step is to determine the number of reads that align
to each predetermined bins on the reference genome.
% bin boundaries
The bin boundaries were determined so that the number of uniquely
mappable regions in the bins are approximately the same across all the
bins (and so the absolute width of each bin varies, with larger bins in
regions of the genome that contain repetitive sequences and smaller in
regions that mostly contain unique sequences).

% problematic regions
Further, there are certain regions of the genome that are known to be
``problematic'' that have an unusually high number of unique reads
aligning to them \citep{amemiya2019encode}. In addition to the ambiguously
mapping regions of the genome, these problematic regions are excluded as
well when determining the bin boundaries.

%
\prog{getBinCounts.py} takes as input a
\href{https://genome.ucsc.edu/FAQ/FAQformat.html#format1}{bed} file
containing the genomic coordinates of the bin boundaries, a bed file
containing the genomic coordinates of all the ambiguously mapping
regions and the problematic regions, and the sam file containing only
the forward reads. For each alignment in the sam file, it is discarded
if it is aligned to ambiguous or problematic regions, otherwise the
count of the bin it belongs to is incremented by one.
% cmd
\begin{verbatim}
$ getBinCounts.py -i sample_fwd.bam \ 
    -b {hg19/hg38}_{bins}_gz_enc_bins.bed \
    -d {hg19/hg38}_150bp_dz_enc.bed -o sample_5k_counts.bed \
    -v > sample_5k_counts_stats.txt
\end{verbatim}

% data
\prog{getBinCounts.py} is located in the
\href{https://github.com/rishvanth-kp/cna_utils/tree/master/scripts}{scripts}
directory and the bed files are located in the
\href{https://github.com/rishvanth-kp/cna_utils/tree/master/data}{data}
directory in the GitHub repository. 
%
For example, to generate profiles for hg38 at 5000 bin resolution, use
\prog{-b hg38\_5k\_gz\_enc\_bins.bed} and 
\prog{-d hg38\_150bp\_dz\_enc.bed}.
%
For generating profiles for hg19 or at other bin resolutions, use
the appropriate bed files.

% output
The output is a bed file that contains the genomic coordinates of the
bin boundaries with an added fourth column containing the counts for
that bin. The stats file provides some basic statistics on the number of
input alignment, the number of filtered alignment, and the number of
alignment that is used for copy number profiling.

%%%%%%%%%%%%%%%%%%%%%%%%%%%%%%%%%%%%%%%%%%%%%%%%%%%%%%%%%%%%%%%%%%%%%%%%%%%%%%%%
\subsection{Generate copy number profiles}
The final step is to convert the bin counts into bin ratios and segmented
copy number profiles. The counts are normalized to account for the
differences in the number of reads sequenced between samples, LOWESS
smoothed to account for the differences in GC content between bins, and
segmented using circular binary segmentation.

% input
\prog{cnvProfile.R} takes as input the bin counts, a bed file containing
the GC content of each bin, and an optional bed file containing the ``bad
bins.''
% cmd
\begin{verbatim}
$ cnvProfile.R -b sample_5k_counts.bed 
    -g {hg19/hg38}_{bins}_gz_enc_gc.bed \
    -e {optional:bad_bins_bed} \
    -o {output_dir} -n {sample_name} -v
\end{verbatim}

% output
The output is a pdf file containing a copy number profile, and two tab
separated files containing all the information related to the copy number
profile such as the bin ratios and the segmented values. One of these
file contains the same number of rows as the number of bins used, and
the other file is shorter version that contains one row for each copy
number segment. When the bad bins are specified, another set of these
three files are generated with the bad bins removed.
%
The these files can be used as an input to any downstream
applications such as generating copy number heatmaps.


%%%%%%%%%%%%%%%%%%%%%%%%%%%%%%%%%%%%%%%%%%%%%%%%%%%%%%%%%%%%%%%%%%%%%%%%%%%%%%%%
%%%%%%%%%%%%%%%%%%%%%%%%%%%%%%%%%%%%%%%%%%%%%%%%%%%%%%%%%%%%%%%%%%%%%%%%%%%%%%%%
%%%%%%%%%%%%%%%%%%%%%%%%%%%%%%%%%%%%%%%%%%%%%%%%%%%%%%%%%%%%%%%%%%%%%%%%%%%%%%%%
% \section{Detmining the bin boundaries}


%%%%%%%%%%%%%%%%%%%%%%%%%%%%%%%%%%%%%%%%%%%%%%%%%%%%%%%%%%%%%%%%%%%%%%%%%%%%%%%%
%%%%%%%%%%%%%%%%%%%%%%%%%%%%%%%%%%%%%%%%%%%%%%%%%%%%%%%%%%%%%%%%%%%%%%%%%%%%%%%%
%%%%%%%%%%%%%%%%%%%%%%%%%%%%%%%%%%%%%%%%%%%%%%%%%%%%%%%%%%%%%%%%%%%%%%%%%%%%%%%%
\bibliographystyle{plainnat}
\bibliography{cnv_pipeline_manual}


\end{document}

